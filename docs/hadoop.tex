\documentclass[11pt]{article}
\usepackage{CJK}
\usepackage{listings}
\begin{CJK}{UTF8}{gkai}
%Gummi|063|=)
\title{\textbf{hadoop相关环境配置}}
\author{闫航}
\date{\today}

\begin{document}

\maketitle
\tableofcontents

\section{hadoop单节点配置}
官方文档里Single Node Setup一节里已经详细地介绍了配置的步骤,照做即可。下面着重讨论以下配置过程中的一些问题。
\begin{enumerate}
\item{\textbf{Java环境配置}} \\
一般的LINUX法相版上都预装了openjdk,足够所用,但hadoop1.0.3建议使用$JAVA^{TM}$1.6.x,所以最好从
Oracle官网上去下载相应的版本。如果下载的是二进制包,选择一个合适的目录直接解压即可。相关命令如下:\\
\begin{lstlisting}
     # cp jdk-*-bin /usr/lib/jvm 
     # chmod a+x jdk-*-bin 
     # ./jdk-*-bin 
     
\end{lstlisting}	
然后需设置JAVA\_HOME,CLASSPATH,PATH这三个环境变量。以在用户Home目录下的.bashrc文件里设置为例,针对上面的操作相应的设置为: \\
\begin{lstlisting}
export JAVA_HOME=/usr/lib/jvm/jdk-*
export CLASSPATH=$JAVA_HOME/lib:$JAVA_HOME/jre/lib:$CLASSPATH
export PATH=$JAVA_HOME/bin:$JAVA_HOME/jre/bin:$PATH
\end{lstlisting}
然后:\\
\begin{lstlisting}
. .bashrc
\end{lstlisting}
使设置的变量生效。
相应地,hadoop的配置文件里也要写明JAVA\_HOME,将conf/hadoop-env.sh里设置JAVA\_HOME的行注释掉,设置与上面相同的值即可。
\item{\textbf{SSH设置}} \\
大多数发行版里都已安装SSH服务,daemon程序名为sshd,即使没有,从源里安装也是很简单的。启动方式为:\\
	\textbf{/etc/init.d/sshd start} \\
默认情况下登录都需要密码,安装官方教程上所说方法即可配置成无密码登录。
\item{\textbf{主机名设置}} \\
hadoop文件系统格式化时注意输出的信息,在最后如果显示有关主机名的问题,很可能就是电脑上主机名设置不当造成的。这会造成hadoop集群的很多问题,查看日志即可得知。所以务必保证在格式化之前保证/etc/hostname,
/etc/hosts,以及hostname命令显示的结果,还有shell提示符的结果一致。
\item{\textbf{hadoop.tmp.dir路径设置}} \\
可以在配置文件conf/core-site.xml里更改其值,示例:\\
\begin{lstlisting}[language=XML]
<property>  
   <name>hadoop.tmp.dir</name>  
   <value>/home/had/hadoop/data</value>  
 <description>A base for other temporary directories.</description>
</property>
\end{lstlisting}
\end{enumerate}

\section{ganglia配置}
\begin{enumerate}
\item{\textbf{安装}} \\
两种安装方式,一种是源码,依赖关系较多,编译着较为麻烦。所以尽量用源里面的软件包。不仅能自动解决依赖关系,而且自动生成了启动脚本,不易出错。Ubuntu10.04源里已提供了相应软件包,直接安装即可。安装后,可用如下方法启动:\\
\textbf{/etc/init.d/ganglia-monitor start} \\ 
\textbf{/etc/init.d/ganglia-monitor start} \\
\item{\textbf{web前端配置}} \\
不管是源码安装,还是在源里面安装,ganglia都提供了一个用于web显示的文件夹,源里安装的话是在/usr/share/ganglia-webfrontend里,将这个文件夹copy至apache网站的一个目录下,即可在网页进行访问,
示例如下: \\
\textbf{cp -r /usr/share/ganglia-webfrontend /var/www/} \\
\textbf{chown -r www-data /var/www/ganglia-webfrontend } \\
\textbf{ chgrp -r www-data /var/www/ganglia-webfrontend } \\
然后即可用过URL \textit{http://127.0.0.1/ganglia-webfrontend/}访问
\end{enumerate}

\section{ganglia监控hadoop配置}
\begin{enumerate}
\item{\textbf{ganglia方面}} \\
主要是修改/etc/ganglia/gmetad.conf文件,添加数据源,格式为:
\begin{lstlisting}
data_source "my cluster" [polling interval] address1:port ...
\end{lstlisting}
单节点情况下,只需添加一个数据源即可,若没有写端口,则默认为8649,所以可以这样写:
\begin{lstlisting}
data_source "hap" localhost
\end{lstlisting}
\item{\textbf{hadoop方面}} \\
hadoop1.0.3版本的ganglia的配置文件为conf/haoop-metrics2.properties,需要做的主要是依据软件版本注释掉相应选项即可。示例:\newpage
\begin{lstlisting}[frame=shadowbox]
*.sink.file.class=
org.apache.hadoop.metrics2.sink.FileSink

# for Ganglia 3.1 support
*.sink.ganglia.class=org.apache.hadoop.\
metrics2.sink.ganglia.GangliaSink31
*.sink.ganglia.period=10

# default for supportsparse is false
*.sink.ganglia.supportsparse=true
*.sink.ganglia.slope=jvm.metrics.gcCount= 
zero,jvm.metrics.memHeapUsedM=both
*.sink.ganglia.dmax=
jvm.metrics.threadsBlocked=
70,jvm.metrics.memHeapUsedM=40

namenode.sink.ganglia.servers=239.2.11.71:8649
datanode.sink.ganglia.servers=239.2.11.71:8649
jobtracker.sink.ganglia.servers=239.2.11.71:8649
tasktracker.sink.ganglia.servers=239.2.11.71:8649
maptask.sink.ganglia.servers=239.2.11.71:8649
reducetask.sink.ganglia.servers=239.2.11.71:8649
\end{lstlisting}

\end{enumerate}

\section{desktop配置}
将解压后的文件夹放至webapps/datanode/目录下,然后通过 \\
\textit{http://localhost:50075/desktop/}访问即可。
搜索项用到了mysql的JDBC驱动,所以需将其放置于hadoop的lib目录下,从源里下载libmysql-java包
即可下载到相应的jar文件。

\section{FileSearch配置}
\begin{enumerate}
\item{\textbf{tomcat}} \\
将FileSearch目录copy至tomcat6网站的根目录即可,在ubuntu上为/var/lib/wepapps/ROOT.
URL:\textit{http://localhost:8080/FileSearch/work/index.jsp}
\item{\textbf{mysql}} \\
暂未改动InverseIndex的源码,所以数据库需做一些修改。SQL:
\begin{lstlisting}[language=SQL]
CREATE DATABASE InverseIndex;
use InverseIndex;

CREATE TABLE TableConf(
fontTable varchar(32),
backTable varchar(32));

CREATE TABLE InverseIndex(
keyword varchar(32),
content varchar(1024));

CREATE TABLE InverseIndex0(
keyword varchar(32),
content varchar(1024));
\end{lstlisting}

INSERT INTO TableConf VALUES
('InverseIndex,InverseIndex0');

\item{\textbf{运行}} \\
\textbf{hadoop jar InverseIndex2.jar "inputpath"}
\end{enumerate}

\section{eclipse环境配置}
将hadoop-eclipse-plugin-1.0.3.jar放至eclipse的plugin目录下即可。
\end{CJK}
\end{document}
